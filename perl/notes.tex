\title{Perl Notes}
%\author{}
\date{\today}

\documentclass[12pt]{article}

\begin{document}
\maketitle

\begin{enumerate}
\item Perl Built-in Warnings 
\begin{verbatim}
#!/usr/bin/perl
use warnings; 
use diagnostics; 
\end{verbatim}
Or
\begin{verbatim}
\$ perl -w program 
\end{verbatim}

\item Single-Quoted and Double-Quoted String

if you wish to use a backslash escape (like $\backslash n$ to mean a newline character), you'll need to use the double quotes. 

\item String Operators 
\begin{enumerate}
\item concatenate, or join, string values: . Operator. 
\item string repetition operator: single lowercase letter x 
\end{enumerate}

\item Automatic Conversion Between Numbers and Strings 

\item Getting User Input 
Each time you use $\textless STDIN\textgreater$ in a place where Perl expects a scalar value, Perl reads the 
next complete text line from standard input (up to the first newline) 
chomp() removes $\backslash n$ 


\item String comparison operators
\begin{enumerate}
\item qual: eq
\item Not equal: ne
\item Less than: lt
\item Greater than: gt
\item Less than or equal to le
\item Greater than or equal to: ge
\end{enumerate}

\item 
Special Array Indices
the last element index: \$\#name
1 (the last element), 2 (the middle element)

List Literals:
a list of comma separated values enclosed in parentheses.
(1, 2, 3)

.. range operator
(1..5) \# same as (1, 2, 3, 4, 5)
(5..1) \# empty list; .. only counts "uphill"

The qw Shortcut
qw( fred barney betty wilma dino ) # same as above, but less typing
qw stands for quoted words or quoted by whitespace,
The previous two examples have used parentheses as the delimiter, but Perl actually lets you choose any punctuation character as the delimiter. 
qw! fred barney betty wilma dino !
qw/ fred barney betty wilma dino /
qw# fred barney betty wilma dino # # like in a comment!

List Assignment
\begin{verbatim}
($fred, $barney, $dino) = ("flintstone", "rubble", undef);
\end{verbatim}

swap:
\begin{verbatim}
($fred, $barney) = ($barney, $fred); # swap those values
($betty[0], $betty[1]) = ($betty[1], $betty[0]);
\end{verbatim}

\begin{verbatim}
($rocks[0], $rocks[1], $rocks[2], $rocks[3]) = qw/talc mica feldspar quartz/;
@rocks is all of the rocks.
@rocks  = qw/ bedrock slate lava /;
@tiny   = ( );                       # the empty list
@giant  = 1..1e5;                    # a list with 100,000 elements
@stuff  = (@giant, undef, @giant);   # a list with 200,001 elements
$dino   = "granite";
@quarry = (@rocks, "crushed rock", @tiny, $dino);
\end{verbatim}

The pop and push Operators: do things to the end of an array
\begin{verbatim}
$fred = pop(@array);
push(@array, 0);
\end{verbatim}

The shift and unshift Operators: perform the corresponding actions on
the start of the array

splice operator: remove or add elements to the middle


\end{enumerate}


%\bibliographystyle{abbrv}
%\bibliography{main}


\end{document}





